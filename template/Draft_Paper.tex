\documentclass[12pt,a4paper]{article}

\usepackage{url}
\usepackage{appendix}
\usepackage[british]{babel}
\usepackage{amsmath}
\usepackage{hyperref}
\usepackage{graphicx}
\numberwithin{equation}{section}
\numberwithin{figure}{section}
\numberwithin{table}{section}

\usepackage[round]{natbib}
\bibliographystyle{natbib}
\def\bibsection{\section*{References}}

% Wrap around which gives all figures included the [H] command, or places it "here". This can be tedious to code in Rmarkdown.
\usepackage{float}
\let\origfigure\figure
\let\endorigfigure\endfigure
\renewenvironment{figure}[1][2] {
    \expandafter\origfigure\expandafter[H]
} {
    \endorigfigure
}

\let\origtable\table
\let\endorigtable\endtable
\renewenvironment{table}[1][2] {
    \expandafter\origtable\expandafter[H]
} {
    \endorigtable
}

\bibpunct[:]{(}{)}{;}{a}{,}{,}

\renewcommand{\baselinestretch}{1.5}

\begin{document}

\begin{titlepage}

\begin{center}
{\Huge \bf Hanjo's Rmarkdown Template}\\
\today\\
A Student (student number 15290743)\\
{\tt 15290743@uct.co.za}
\end{center}

\begin{abstract}
Abstract to be written here. The abstract should not be too long and
should provide the reader with a good understanding what you are writing
about. Academic papers are not like novels where you keep the reader in
suspense. To be effective in getting others to read your paper, be as
open and concise about your findings here as possible. Ideally, upon
reading your abstract, the reader should feel he / she must read your
paper in entirety.
\noindent
Keywords: literature, methodology, research,   results, structure, template.
\end{abstract}
\end{titlepage}

\pagenumbering{arabic}

\section{\texorpdfstring{Introduction
\label{Introduction}}{Introduction }}\label{introduction}

References are to be made as follows: Fama and French (1997, 33) and
Grinold and Kahn (2000) Such authors could also be referenced in
brackets (Grinold and Kahn 2000) and together (Fama and French 1997 \&
Grinold and Kahn (2000)). Source the reference code from
scholar.google.com by clicking on ``cite'' below article name. Then
select BibTeX at the bottom of the Cite window, and proceed to copy and
paste this code into your ref.bib file, located in the directory's Tex
folder. Open this file in Rstudio for ease of management, else open it
in your preferred Tex environment. Add and manage your article details
here for simplicity - once saved, it will self-adjust in your paper.

\begin{quote}
I suggest renaming the top line after @article, as done in the template
ref.bib file, to something more intuitive for you to remember. Do not
change the rest of the code. Also, be mindful of the fact that bib
references from google scholar may at times be incorrect. Reference
Latex forums for correct bibtex notation.
\end{quote}

To reference a section, you have to set a label using
``\textbackslash{}label'' in R, and then reference it in-text as e.g.:
section \ref{Data}.

Writing in Rmarkdown is surprizingly easy - see
\href{https://www.rstudio.com/wp-content/uploads/2015/03/rmarkdown-reference.pdf}{this
website} cheatsheet for a summary on writing Rmd writing tips.

\section{\texorpdfstring{Data \label{Data}}{Data }}\label{data}

Discussion of data should be thorough with a table of statistics and
ideally a figure.

In your tempalte folder, you will find a Data and a Code folder. In
order to keep your data files neat, store all of them in your Data
folder. Also, I strongly suggest keeping this Rmd file for writing and
executing commands, not writing out long pieces of data-wrangling. In
the example below, I simply create a ggplot template for scatter plot
consistency. I suggest keeping all your data in a data folder.

\begin{figure}[H]

{\centering \includegraphics{Draft_Paper_files/figure-latex/Figure1-1} 

}

\caption{Caption Here \label{Figure1}}\label{fig:Figure1}
\end{figure}

To reference the plot above, add a ``\textbackslash{}label'' after the
caption in the chunk heading, as done above. Then reference the plot as
such: As can be seen, figure \ref{Figure1} is excellent. The nice thing
now is that it correctly numbers all your figures (and sections or
tables) and will update if it moves. The links are also dynamic.

I very strongly suggest using ggplot2 (ideally in combination with
dplyr) using the ggtheme package to change the themes of your figures.

Also note the information that I have placed above the chunks in the
code chunks for the figures. You can edit any of these easily - visit
the Rmarkdown webpage for more information.

Here follows another figure from built-in ggplot2 data:

\begin{figure}[H]

{\centering \includegraphics{Draft_Paper_files/figure-latex/figure2-1} 

}

\caption{Diamond Cut Plot \label{lit}}\label{fig:figure2}
\end{figure}

\section{Methodology}\label{methodology}

\subsection{Subsection}\label{subsection}

Ideally do not overuse subsections. It equates to bad writing.\footnote{This
  is an example of a footnote by the way. Something that should also not
  be overused.}

\subsection{Math section}\label{math-section}

Equations should be written as such:

\begin{align} 
\beta = \sum_{i = 1}^{\infty}\frac{\alpha^2}{\sigma_{t-1}^2} \label{eq1} \\ 
\int_{x = 1}^{\infty}x_{i} = 1 \notag
\end{align}

If you would like to see the equations as you type in Rmarkdown, use \$
symbols instead (see this for yourself by adjusted the equation):

\[
\beta = \sum_{i = 1}^{\infty}\frac{\alpha^2}{\sigma_{t-1}^2} \\ 
\int_{x = 1}^{\infty}x_{i} = 1
\]

\begin{align} 
    y_t &= c + B(L) y_{t-1} + e_t   \label{eq2}    \\ \notag 
    e_t &= H_t^{1/2}  z_t ; \quad z_t \sim  N(0,I_N) \quad \& \quad H_t = D_tR_tD_t \\ \notag
        D_t^2 &= {\sigma_{1,t}, \dots, \sigma_{N,t}}   \\ \notag
        \sigma_{i,t}^2 &= \gamma_i+\kappa_{i,t}  v_{i, t-1}^2 +\eta_i  \sigma_{i, t-1}^2, \quad \forall i \\ \notag
        R_{t, i, j} &= {diag(Q_{t, i, j}}^{-1}) . Q_{t, i, j} . diag(Q_{t, i, j}^{-1})  \\ \notag
        Q_{t, i, j} &= (1-\alpha-\beta)  \bar{Q} + \alpha  z_t  z_t'  + \beta  Q_{t, i, j} \notag
\end{align}

Noten that in \ref{eq2} I have aligned the equations by the equal signs.
I also want only one tag, and I create spaces using ``quads''.

See if you can figure out how to do complex math using the two examples
provided in \ref{eq1} and \ref{eq2}.

\section{Results}\label{results}

Tables can be included as follows. Use the \emph{xtable} (or kable)
package for tables. Table placement = H implies Latex tries to place the
table Here, and not on a new page (there are, however, very many ways to
skin this cat. Luckily there are many forums online!).

To reference calculations \textbf{in text}, \emph{do this:} From table
\ref{tab1} we see the average value of mpg is 20.98.

Including tables that span across pages, use e.g.:

\hfill

\section{Lists}\label{lists}

To add lists, simply using the following notation

\begin{itemize}
\item
  This is really simple

  \begin{itemize}
  \tightlist
  \item
    Just note the spaces here - writing in R you have to sometimes be
    pedantic about spaces\ldots{}
  \end{itemize}
\item
  Note that Rmarkdown notation removes the pain of defining
  \LaTeX environments!
\end{itemize}

\section{Conclusion}\label{conclusion}

I hope you find this template useful. Remember, stackoverflow is your
friend - use it to find answers to questions. Feel free to write me a
mail if you have any questions regarding the use of this package feel
free to email me. To cite this package, simply type
citation(``Texevier'') in Rstudio to get the citation for Katzke (2017)
(Note that uncited references in your bibtex file will not be included
in References).

\section*{References}\label{references}
\addcontentsline{toc}{section}{References}

\hypertarget{refs}{}
\hypertarget{ref-fama1997}{}
Fama, Eugene F, and Kenneth R French. 1997. ``Industry Costs of
Equity.'' \emph{Journal of Financial Economics} 43 (2). Elsevier:
153--93.

\hypertarget{ref-grinold2000}{}
Grinold, Richard C, and Ronald N Kahn. 2000. ``Active Portfolio
Management.'' McGraw Hill New York, NY.

\hypertarget{ref-Texevier}{}
Katzke, N.F. 2017. \emph{Texevier: Package to Create Elsevier Templates
for Rmarkdown}. Stellenbosch, South Africa: Bureau for Economic
Research.

\newpage
\renewcommand{\baselinestretch}{1}
\nocite{*}
\bibliography{}

\end{document}
